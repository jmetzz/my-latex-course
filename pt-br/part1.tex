\documentclass{beamer}

\input{preamble.tex}

\subtitle{Parte 1: Conceitos Básicos}

\begin{document}

%%%%%%%%%%%%%%%%%%%%%%%%%%%%%%%%%%%%%%%%%%%%%%%%%%%%%%%%%%%%%%%%%%%%%%%%%%%%%%%
%%%%%%%%%%%%%%%%%%%%%%%%%%%%%%%%%%%%%%%%%%%%%%%%%%%%%%%%%%%%%%%%%%%%%%%%%%%%%%%
%%%%%%%%%%%%%%%%%%%%%%%%%%%%%%%%%%%%%%%%%%%%%%%%%%%%%%%%%%%%%%%%%%%%%%%%%%%%%
\begin{frame}
\titlepage
\end{frame}

\begin{frame}

\begin{center}
Tradu\c{c}\~ ao por: Jean Metz

\textcolor{gray}{\href{http://www.jean.metzz.org/}{My homepage}}

\vskip 10ex


\includegraphics[height=24pt]{UTFPR-logo}
\end{center}

\end{frame}


%%%%%%%%%%%%%%%%%%%%%%%%%%%%%%%%%%%%%%%%%%%%%%%%%%%%%%%%%%%%%%%%%%%%%%%%%%%%%%%
%%%%%%%%%%%%%%%%%%%%%%%%%%%%%%%%%%%%%%%%%%%%%%%%%%%%%%%%%%%%%%%%%%%%%%%%%%%%%%%
%%%%%%%%%%%%%%%%%%%%%%%%%%%%%%%%%%%%%%%%%%%%%%%%%%%%%%%%%%%%%%%%%%%%%%%%%%%%%%%
\begin{frame}{Por que \LaTeX{}?}
	\begin{itemize}
		\item Ele faz documentos bonitos e bem formatados
			\begin{itemize}
				\item Especialmente com conteúdo matemático
			\end{itemize}
%
		\item Ele foi criado por cientistas, para cientistas
			\begin{itemize}
				\item Há uma grande e ativa comunidade
			\end{itemize}
%
		\item Ele é poderoso --- e você pode estendê-lo
			\begin{itemize}
				\item Existem pacotes para artigos, apresentações, planilhas eletrônicas,  \ldots
			\end{itemize}
	\end{itemize}
\end{frame}

%%%%%%%%%%%%%%%%%%%%%%%%%%%%%%%%%%%%%%%%%%%%%%%%%%%%%%%%%%%%%%%%%%%%%%%%%%%%%%%%
%%%%%%%%%%%%%%%%%%%%%%%%%%%%%%%%%%%%%%%%%%%%%%%%%%%%%%%%%%%%%%%%%%%%%%%%%%%%%%%%
%%%%%%%%%%%%%%%%%%%%%%%%%%%%%%%%%%%%%%%%%%%%%%%%%%%%%%%%%%%%%%%%%%%%%%%%%%%%%%%%
\begin{frame}[fragile]{Como ele funciona?}
\begin{itemize}
\item Você escreve seu documento em \texttt{texto puro} com \cmd{comandos} que descrevem a estrutura e significado do texto.

\item O programa \texttt{latex} processa seu texto e comandos para produzir um documento bem formatado e bonito.
\end{itemize}
\vskip 2ex
\begin{center}
\begin{minted}[frame=single]{latex}
A chuva na Espanha cai \emph{principalmente} 

na plan\'icie.
\end{minted}
\vskip 2ex
\tikz\node[single arrow,fill=gray,font=\ttfamily\bfseries,%
  rotate=270,xshift=-1em]{latex};
\vskip 2ex
\fbox{A chuva na Espanha cai \emph{principalmente} na planície.}
\end{center}
\end{frame}

%%%%%%%%%%%%%%%%%%%%%%%%%%%%%%%%%%%%%%%%%%%%%%%%%%%%%%%%%%%%%%%%%%%%%%%%%%%%%%%
%%%%%%%%%%%%%%%%%%%%%%%%%%%%%%%%%%%%%%%%%%%%%%%%%%%%%%%%%%%%%%%%%%%%%%%%%%%%%%%
%%%%%%%%%%%%%%%%%%%%%%%%%%%%%%%%%%%%%%%%%%%%%%%%%%%%%%%%%%%%%%%%%%%%%%%%%%%%%%%
\begin{frame}[fragile]{Mais exemplos de comandos e suas respectivas saídas \ldots}
	\begin{exampletwoup}
		\begin{itemize}
			\item Ch\'a
			\item Leite
			\item Biscoito
		\end{itemize}
	\end{exampletwoup}
\vskip 2ex
	\begin{exampletwoup}
		\begin{figure}
			\includegraphics{chick}
		\end{figure}
	\end{exampletwoup}
\vskip 2ex
	\begin{exampletwoup}
		\begin{equation}
			\alpha + \beta + 1
		\end{equation}
	\end{exampletwoup}

	\tiny{Image from \url{http://www.andy-roberts.net/writing/latex/importing_images}}
\end{frame}

%%%%%%%%%%%%%%%%%%%%%%%%%%%%%%%%%%%%%%%%%%%%%%%%%%%%%%%%%%%%%%%%%%%%%%%%%%%%%%%
%%%%%%%%%%%%%%%%%%%%%%%%%%%%%%%%%%%%%%%%%%%%%%%%%%%%%%%%%%%%%%%%%%%%%%%%%%%%%%%
%%%%%%%%%%%%%%%%%%%%%%%%%%%%%%%%%%%%%%%%%%%%%%%%%%%%%%%%%%%%%%%%%%%%%%%%%%%%%%%
\begin{frame}[fragile]{Ajuste de atitude}

\begin{itemize}
	\item Use comandos para descrever `o que é', não `como parece'.
	\item Foque no conteúdo.
	\item Deixe que o \LaTeX{} faça o trabalho.
\end{itemize}
\end{frame}

%%%%%%%%%%%%%%%%%%%%%%%%%%%%%%%%%%%%%%%%%%%%%%%%%%%%%%%%%%%%%%%%%%%%%%%%%%%%%%%
%%%%%%%%%%%%%%%%%%%%%%%%%%%%%%%%%%%%%%%%%%%%%%%%%%%%%%%%%%%%%%%%%%%%%%%%%%%%%%%
%%%%%%%%%%%%%%%%%%%%%%%%%%%%%%%%%%%%%%%%%%%%%%%%%%%%%%%%%%%%%%%%%%%%%%%%%%%%%%%
\section{Básico}

%%%%%%%%%%%%%%%%%%%%%%%%%%%%%%%%%%%%%%%%%%%%%%%%%%%%%%%%%%%%%%%%%%%%%%%%%%%%%%%
%%%%%%%%%%%%%%%%%%%%%%%%%%%%%%%%%%%%%%%%%%%%%%%%%%%%%%%%%%%%%%%%%%%%%%%%%%%%%%%
%%%%%%%%%%%%%%%%%%%%%%%%%%%%%%%%%%%%%%%%%%%%%%%%%%%%%%%%%%%%%%%%%%%%%%%%%%%%%%%
\subsection{Começando}
\begin{frame}[fragile]{\insertsubsection}
\begin{itemize}
	\item Um documento \LaTeX{} mínimo:
		\inputminted[frame=single]{latex}{basics.tex}
	
	\item Comandos começam com  \emph{backslash} \keystrokebftt{\bs}.
	\item Todo documento comaça com um comando \cmdbs{documentclass}.
	\item O \emph{argumento} dentro das chaves \keystrokebftt{\{} \keystrokebftt{\}} representam que tipo de documento \LaTeX{} estamos criando: um \bftt{article}.
	\item O símbolo de percentual \keystrokebftt{\%} é usado para marcar o início de \emph{comentários} --- o \LaTeX{} vai ignorar o restante da linha.
\end{itemize}
\end{frame}


%%%%%%%%%%%%%%%%%%%%%%%%%%%%%%%%%%%%%%%%%%%%%%%%%%%%%%%%%%%%%%%%%%%%%%%%%%%%%%%
%%%%%%%%%%%%%%%%%%%%%%%%%%%%%%%%%%%%%%%%%%%%%%%%%%%%%%%%%%%%%%%%%%%%%%%%%%%%%%%
%%%%%%%%%%%%%%%%%%%%%%%%%%%%%%%%%%%%%%%%%%%%%%%%%%%%%%%%%%%%%%%%%%%%%%%%%%%%%%%
\begin{frame}[fragile]{\insertsubsection{} com \wllogo}
	\begin{itemize}
		\item write\LaTeX{} é um \textit{website} para escrita de documentos em \LaTeX.
		\item Ele `compila' seu código \LaTeX{} automaticamente para te mostrar o resultado.
			\vskip 2em
			\begin{center}
				\fbox{\href{\wlnewdoc{basics.tex}}{Clique aqui para abrir um exemplo de documento no \wllogo{}}}
				\\[1ex]\scriptsize{}
				Ou v\'a para essa URL: \url{http://bit.ly/WU0bMU}\\
				Para melhores resultados, por favor use \href{http://www.google.com/chrome}{Google Chrome} ou uma vers\~ao recente do \href{http://www.mozilla.org/en-US/firefox/new/}{FireFox}.
			\end{center}
		\vskip 2ex
		\item Conforme veremos os próximos slides, teste os exemplos os digitando no documento de exemplo no write\LaTeX.
		\item \textbf{Não agora. Você deve testá-los conforme vamos passando pelos exemplos!}
	\end{itemize}
\end{frame}

%%%%%%%%%%%%%%%%%%%%%%%%%%%%%%%%%%%%%%%%%%%%%%%%%%%%%%%%%%%%%%%%%%%%%%%%%%%%%%%
%%%%%%%%%%%%%%%%%%%%%%%%%%%%%%%%%%%%%%%%%%%%%%%%%%%%%%%%%%%%%%%%%%%%%%%%%%%%%%%
%%%%%%%%%%%%%%%%%%%%%%%%%%%%%%%%%%%%%%%%%%%%%%%%%%%%%%%%%%%%%%%%%%%%%%%%%%%%%%%
\subsection{Compondo o Texto}
\begin{frame}[fragile]{\insertsubsection{}}
	\small
	\begin{itemize}
		\item Digite seu texto entre \cmdbegin{document} e \cmdend{document}.
		\item Para a maior parte, você pode apenas digitar seu texto normalmente.
	\begin{exampletwouptiny}
		Palavras s\~ao separadas por um ou 
		mais espa\c{c}os.
		
		Par\'agrafos s\~ao separados por uma 
		ou mais linhas em branco.
	\end{exampletwouptiny}
	\item Espaços no arquivo fonte são truncados no arquivo de saída.
		\begin{exampletwouptiny}
			A     chuva      na Espanha
			cai principalmente na plan\'icie.
		\end{exampletwouptiny}
	\end{itemize}
\end{frame}

%%%%%%%%%%%%%%%%%%%%%%%%%%%%%%%%%%%%%%%%%%%%%%%%%%%%%%%%%%%%%%%%%%%%%%%%%%%%%%%
%%%%%%%%%%%%%%%%%%%%%%%%%%%%%%%%%%%%%%%%%%%%%%%%%%%%%%%%%%%%%%%%%%%%%%%%%%%%%%%
%%%%%%%%%%%%%%%%%%%%%%%%%%%%%%%%%%%%%%%%%%%%%%%%%%%%%%%%%%%%%%%%%%%%%%%%%%%%%%%
\begin{frame}[fragile]{\insertsubsection{}: Cuidado}
\small
	\begin{itemize}
	\item Aspas são um pouco complicadas: use crase \keystroke{\`{}} à esquerda e apóstrofe \keystroke{\'{}} à direita.
		\begin{exampletwouptiny}
			Aspas simples: `texto'.
	
			Aspas duplas: ``texto''.
		\end{exampletwouptiny}
	
	\item Em \LaTeX{} alguns caracteres comuns são especiais :\\[1ex]
		\begin{tabular}{cl}
			\keystrokebftt{\%} & símbolo percentual              \\
			\keystrokebftt{\#} & cerquilha  \\
			\keystrokebftt{\&} & e-comercial                 \\
			\keystrokebftt{\$} & cifrão               \\
		\end{tabular}
	\item Se você apenas digitá-los, terá um erro como resultado. Se você quer que um desses caracteres apareça na saída, terá que usar um caractere de \emph{escape} como prefixo: a barra invertida \keystrokebftt{$\backslash$}.
		\begin{exampletwoup}
			\$\%\&\#!
		\end{exampletwoup}
	\end{itemize}
\end{frame}

%%%%%%%%%%%%%%%%%%%%%%%%%%%%%%%%%%%%%%%%%%%%%%%%%%%%%%%%%%%%%%%%%%%%%%%%%%%%%%%
%%%%%%%%%%%%%%%%%%%%%%%%%%%%%%%%%%%%%%%%%%%%%%%%%%%%%%%%%%%%%%%%%%%%%%%%%%%%%%%
%%%%%%%%%%%%%%%%%%%%%%%%%%%%%%%%%%%%%%%%%%%%%%%%%%%%%%%%%%%%%%%%%%%%%%%%%%%%%%%
\begin{frame}[fragile]{Tratando Erros}
	\begin{itemize}
		\item O compilador \LaTeX{} pode se confundir quando estiver tentanto 
		compilar o seu documento. Se isso acontecer, ele para e apresenta um erro, 
		o qual você deve corrigir antes que ele possa produzir o arquivo de saída.
		
		\item Por exemplo, se voc\^e digitar erroneamente \cmdbs{emph} como \cmdbs{meph},
		o \LaTeX{} vai parar com o erro ``undefined control sequence'', pois ``meph'' não 
		é um dos comandos conhecidos.
	\end{itemize}
	
	\begin{block}{Dicas em caso de erros}
		\begin{enumerate}
			\item \textit{Don't panic}! Erros acontecem.
			\item Corrija assim que eles aparecerem --- se o que voc\^e acabou de digitar causou um erro, voc\^e pode \textit{debuggar} a partir desse ponto.
			\item Se existem m\'ultiplos erros, comece com o primeiro --- a causa pode estar acima dele :(
		\end{enumerate}
	\end{block}
\end{frame}

%%%%%%%%%%%%%%%%%%%%%%%%%%%%%%%%%%%%%%%%%%%%%%%%%%%%%%%%%%%%%%%%%%%%%%%%%%%%%%%
%%%%%%%%%%%%%%%%%%%%%%%%%%%%%%%%%%%%%%%%%%%%%%%%%%%%%%%%%%%%%%%%%%%%%%%%%%%%%%%
%%%%%%%%%%%%%%%%%%%%%%%%%%%%%%%%%%%%%%%%%%%%%%%%%%%%%%%%%%%%%%%%%%%%%%%%%%%%%%%
\begin{frame}[fragile]{Compondo o Texto - Exerc\'icio 1}

	\begin{block}{Digite isso em \LaTeX:
	\footnote{\url{http://en.wikipedia.org/wiki/Economy_of_the_United_States}}}
		In March 2006, Congress raised that ceiling an additional \$0.79 trillion to
		\$8.97 trillion, which is approximately 68\% of GDP. As of October 4, 2008, the
		``Emergency Economic Stabilization Act of 2008'' raised the current debt ceiling
		to \$11.3 trillion.
	\end{block}
	\vskip 2ex
	\begin{center}
		\fbox{\href{\wlnewdoc{basics-exercise-1.tex}}{%
		Clique no \wllogo{} para abrir esse exerc\'icio}}
	\end{center}
	
	\begin{itemize}
		\item Dica: cuidado com os caracteres com significado especial!
		\item Uma vez que você tenha tentando, 
			\fbox{\href{\wlnewdoc{basics-exercise-1-solution.tex}}{%
		clique aqui para ver a soluç\~ao}}.
	\end{itemize}
\end{frame}

%%%%%%%%%%%%%%%%%%%%%%%%%%%%%%%%%%%%%%%%%%%%%%%%%%%%%%%%%%%%%%%%%%%%%%%%%%%%%%%
%%%%%%%%%%%%%%%%%%%%%%%%%%%%%%%%%%%%%%%%%%%%%%%%%%%%%%%%%%%%%%%%%%%%%%%%%%%%%%%
%%%%%%%%%%%%%%%%%%%%%%%%%%%%%%%%%%%%%%%%%%%%%%%%%%%%%%%%%%%%%%%%%%%%%%%%%%%%%%%
\subsection{Compondo Equaç\~oes Matem\'aticas}
\begin{frame}[fragile]{\insertsubsection{}: Cifr\~ao}
	\begin{itemize}
		\item Por que o caracteter cifr\~ao \keystrokebftt{\$} \'e especial? Porque usamos esse caractere para marcar elementos matem\'aticos no texto.\\[1ex]
	\begin{exampletwouptiny}
		% ruim:
		Considere a e b inteiros positivos
		distintos, e considere c = a - b + 1
		
		% melhor:
		Considere $a$ e $b$ inteiros positivos
		distintos, e considere $c = a - b + 1$
	\end{exampletwouptiny}
	\item Sempre use o cifr\~ao em pares --- um para começar e outro para finalizar o conte\'udo matem\'atico.
	
	\item \LaTeX{} trata espaços automaticamente; ele ignora seus espaços.

	\begin{exampletwouptiny}
		Seja $y=mx+b$ \ldots
		
		Seja $y = m x + b$ \ldots
	\end{exampletwouptiny}
	\end{itemize}
\end{frame}

%%%%%%%%%%%%%%%%%%%%%%%%%%%%%%%%%%%%%%%%%%%%%%%%%%%%%%%%%%%%%%%%%%%%%%%%%%%%%%%
%%%%%%%%%%%%%%%%%%%%%%%%%%%%%%%%%%%%%%%%%%%%%%%%%%%%%%%%%%%%%%%%%%%%%%%%%%%%%%%
%%%%%%%%%%%%%%%%%%%%%%%%%%%%%%%%%%%%%%%%%%%%%%%%%%%%%%%%%%%%%%%%%%%%%%%%%%%%%%%
\begin{frame}[fragile]{\insertsubsection{}: Notação}
	\begin{itemize}
		\item Use circunflexo \keystrokebftt{\^} para sobrescritos e \textit{underscore} \keystrokebftt{\_} para subscritos.
			\begin{exampletwouptiny}
				$y = c_2 x^2 + c_1 x + c_0$
			\end{exampletwouptiny}
			\vskip 2ex
		
		\item Use chaves \keystrokebftt{\{} \keystrokebftt{\}} para agrupar sobrescritos e subscritos.
			\begin{exampletwouptiny}
				$F_n = F_n-1 + F_n-2$     % oops!
				
				$F_n = F_{n-1} + F_{n-2}$ % ok!
			\end{exampletwouptiny}
		\vskip 2ex
		
		\item Existem comandos para letras do alfabeto Grego e notação comum.
			\begin{exampletwouptiny}
				$\mu = A e^{Q/RT}$
				
				$\Omega = \sum_{k=1}^{n} \omega_k$
			\end{exampletwouptiny}
	\end{itemize}
\end{frame}

%%%%%%%%%%%%%%%%%%%%%%%%%%%%%%%%%%%%%%%%%%%%%%%%%%%%%%%%%%%%%%%%%%%%%%%%%%%%%%%
%%%%%%%%%%%%%%%%%%%%%%%%%%%%%%%%%%%%%%%%%%%%%%%%%%%%%%%%%%%%%%%%%%%%%%%%%%%%%%%
%%%%%%%%%%%%%%%%%%%%%%%%%%%%%%%%%%%%%%%%%%%%%%%%%%%%%%%%%%%%%%%%%%%%%%%%%%%%%%%
\begin{frame}[fragile]{\insertsubsection{}: Mostrando Equaç\~oes}
\begin{itemize}
\item Se for uma equação grande e assustadora, \emph{mostre-a} em uma linha ``própria'' usando o comando
\cmdbegin{equation} e \cmdend{equation}.
\\
[2ex]
\begin{exampletwouptiny}
	As ra\'izes de um equa\c{c}\~ao 
	quadrada s\~ao dadas por 
	\begin{equation}
	x = \frac{-b \pm \sqrt{b^2 - 4ac}}
	         {2a}
	\end{equation}
	onde $a$, $b$ e $c$ s\~ao \ldots
\end{exampletwouptiny}
\vskip 1em
{\scriptsize Atenção: \LaTeX{} ignora espaços em elementos matemáticos, mas não aceita linhas em branco em equações --- não coloque linhas em brano nas suas equações.}
\end{itemize}
\end{frame}

%%%%%%%%%%%%%%%%%%%%%%%%%%%%%%%%%%%%%%%%%%%%%%%%%%%%%%%%%%%%%%%%%%%%%%%%%%%%%%%
%%%%%%%%%%%%%%%%%%%%%%%%%%%%%%%%%%%%%%%%%%%%%%%%%%%%%%%%%%%%%%%%%%%%%%%%%%%%%%%
%%%%%%%%%%%%%%%%%%%%%%%%%%%%%%%%%%%%%%%%%%%%%%%%%%%%%%%%%%%%%%%%%%%%%%%%%%%%%%%
\begin{frame}[fragile]{Ambientes}
	\begin{itemize}
		\item \bftt{equation} é um  \emph{ambiente} --- um contexto.
		\item Um mesmo comando pode produzir sa\'idas distintas em diferentes contextos.
		
		\begin{exampletwouptiny}
			Podemos escrever
			$ \Omega = \sum_{k=1}^{n} \omega_k $
			no corpo do texto, ou podemos escrever
			\begin{equation}
			  \Omega = \sum_{k=1}^{n} \omega_k
			\end{equation}
			para mostrar a equa\c{c}\~ao.
		\end{exampletwouptiny}
		\vskip 2ex
		\item Observe como o comando $\Sigma$ \'e maior dentro do ambiente \bftt{equation}, e como os sub-escritos e super-escritos aparecem em posi\c{c}\~oes diferentes, ainda que sejam o mesmo comando. 
		\vskip 1em
		{\scriptsize De fato, poder\'iamos ter escrito \bftt{\$...\$} como
		\cmdbegin{math}\bftt{...}\cmdend{math}.}
	\end{itemize}
\end{frame}

%%%%%%%%%%%%%%%%%%%%%%%%%%%%%%%%%%%%%%%%%%%%%%%%%%%%%%%%%%%%%%%%%%%%%%%%%%%%%%%
%%%%%%%%%%%%%%%%%%%%%%%%%%%%%%%%%%%%%%%%%%%%%%%%%%%%%%%%%%%%%%%%%%%%%%%%%%%%%%%
%%%%%%%%%%%%%%%%%%%%%%%%%%%%%%%%%%%%%%%%%%%%%%%%%%%%%%%%%%%%%%%%%%%%%%%%%%%%%%%
\begin{frame}[fragile]{Ambientes}
	\begin{itemize}
		\item Os comandos \cmdbs{begin} e \cmdbs{end} s\~ao usados para criar muitos ambientes diferentes.
		\vskip 2ex
		
		\item Os ambientes \bftt{itemize} e \bftt{enumerate} s\~ao usados para gerar listas.
		\begin{exampletwouptiny}
			% para marcadores com s\'imbolos
			\begin{itemize} 
			    \item Biscoitos
			    \item Ch\'a
			\end{itemize}
			
			% para marcadores num\'ericos
			\begin{enumerate} 
			    \item Biscoitos
			    \item Ch\'a
			\end{enumerate}
		\end{exampletwouptiny}
	\end{itemize}
\end{frame}

%%%%%%%%%%%%%%%%%%%%%%%%%%%%%%%%%%%%%%%%%%%%%%%%%%%%%%%%%%%%%%%%%%%%%%%%%%%%%%%
%%%%%%%%%%%%%%%%%%%%%%%%%%%%%%%%%%%%%%%%%%%%%%%%%%%%%%%%%%%%%%%%%%%%%%%%%%%%%%%
%%%%%%%%%%%%%%%%%%%%%%%%%%%%%%%%%%%%%%%%%%%%%%%%%%%%%%%%%%%%%%%%%%%%%%%%%%%%%%%
\begin{frame}[fragile]{Pacotes}

	\begin{itemize}
		\item Todos os comandos e ambientes usados at\'e agora est\~ao presentes 
		na distribui\c{c}\~ao b\'asica do \LaTeX.
		
		\item \emph{Pacotes} são bibliotecas com comandos e ambientes extras. Existem centenas de pacotes disponíveis (\textit{free}).
		
		\item É necessário carregar todos os pacotes de interesse usando o comando 
		\cmdbs{usepackage} no \emph{preâmbulo} do documento.
		
		\item Exemplo: \bftt{amsmath} da \textit{American Mathematical Society}.
			\begin{minted}[fontsize=\small,frame=single]{latex}
				\documentclass{article}
				\usepackage{amsmath} % preamble
				\begin{document}
					% agora podemos usar os comandos 
					% definidos em amsmath ...
				\end{document}
			\end{minted}
	\end{itemize}
\end{frame}

%%%%%%%%%%%%%%%%%%%%%%%%%%%%%%%%%%%%%%%%%%%%%%%%%%%%%%%%%%%%%%%%%%%%%%%%%%%%%%%
%%%%%%%%%%%%%%%%%%%%%%%%%%%%%%%%%%%%%%%%%%%%%%%%%%%%%%%%%%%%%%%%%%%%%%%%%%%%%%%
%%%%%%%%%%%%%%%%%%%%%%%%%%%%%%%%%%%%%%%%%%%%%%%%%%%%%%%%%%%%%%%%%%%%%%%%%%%%%%%
\begin{frame}[fragile]{\insertsubsection{}: Exemplos com \bftt{amsmath}}
\vskip 3ex
	\begin{itemize}
		\item Use \bftt{equation*} (``equation-star'') para remover a numeração das equações.
			\begin{exampletwouptiny}
				\begin{equation*}
				  \Omega = \sum_{k=1}^{n} \omega_k
				\end{equation*}
			\end{exampletwouptiny}
	\end{itemize}
\end{frame}


\begin{frame}[fragile]{\insertsubsection{}: Exemplos com \bftt{amsmath}}
	\begin{itemize}

		\item \LaTeX{} trata letras adjacentes como multiplicação de variáveis, o que nem sempre é o que queremos. \bftt{amsmath} define comandos para muitos operadores matemáticos comuns.
			\begin{exampletwouptiny}
				\begin{equation*} % bad!
				 min_{x,y} (1-x)^2 + 100(y-x^2)^2 
				\end{equation*}
				\begin{equation*} % good!
				\min_{x,y}{(1-x)^2 + 100(y-x^2)^2}
				\end{equation*}
			\end{exampletwouptiny}

		\item Para outros comandos, você pode usar \cmdbs{operatorname}.
			\begin{exampletwouptiny}
				\begin{equation*}
				\beta_i =
				\frac{\operatorname{Cov}(R_i, R_m)}
				     {\operatorname{Var}(R_m)}
				\end{equation*}
			\end{exampletwouptiny}
	\end{itemize}
\end{frame}

%%%%%%%%%%%%%%%%%%%%%%%%%%%%%%%%%%%%%%%%%%%%%%%%%%%%%%%%%%%%%%%%%%%%%%%%%%%%%%%
%%%%%%%%%%%%%%%%%%%%%%%%%%%%%%%%%%%%%%%%%%%%%%%%%%%%%%%%%%%%%%%%%%%%%%%%%%%%%%%
%%%%%%%%%%%%%%%%%%%%%%%%%%%%%%%%%%%%%%%%%%%%%%%%%%%%%%%%%%%%%%%%%%%%%%%%%%%%%%%
\begin{frame}[fragile]{\insertsubsection{}: Exemplos com \bftt{amsmath}}
	\begin{itemize}
	{\small
		\item Alinhando uma sequência de equações ao sinal de igualdade
		\begin{align*}
		(x+1)^3 &= (x+1)(x+1)(x+1) \\
		        &= (x+1)(x^2 + 2x + 1) \\
		        &= x^3 + 3x^2 + 3x + 1
		\end{align*}
		com o ambiente \bftt{align*}.
		
		% for whatever reason, this doesn't play well with the twoup environment
		\begin{minted}[fontsize=\small,frame=single]{latex}
			\begin{align*}
			(x+1)^3 &= (x+1)(x+1)(x+1) \\
			        &= (x+1)(x^2 + 2x + 1) \\
			        &= x^3 + 3x^2 + 3x + 1
			\end{align*}
		\end{minted}
		\item O símbolo \keystrokebftt{\&} separa as colunas esquerda (antes do sinal
		$=$) e direita  (depois do $=$).
		\item Para iniciar uma nova linha, usa-se o duas vezes o símbolo de \textit{back slash}, ou seja \keystrokebftt{\bs\bs}.
	}
	\end{itemize}
\end{frame}


%%%%%%%%%%%%%%%%%%%%%%%%%%%%%%%%%%%%%%%%%%%%%%%%%%%%%%%%%%%%%%%%%%%%%%%%%%%%%%%
%%%%%%%%%%%%%%%%%%%%%%%%%%%%%%%%%%%%%%%%%%%%%%%%%%%%%%%%%%%%%%%%%%%%%%%%%%%%%%%
%%%%%%%%%%%%%%%%%%%%%%%%%%%%%%%%%%%%%%%%%%%%%%%%%%%%%%%%%%%%%%%%%%%%%%%%%%%%%%%
\begin{frame}[fragile]{Compondo o Texto - Exercício 2}

	\begin{block}{Escreva esse texto em \LaTeX:}
		Let $X_1, X_2, \ldots, X_n$ be a sequence of independent and identically
		distributed random variables with $\operatorname{E}[X_i] = \mu$ and
		$\operatorname{Var}[X_i] = \sigma^2 < \infty$, and let
		\begin{equation*}
		S_n = \frac{1}{n}\sum_{i}^{n} X_i
		\end{equation*}
		denote their mean. Then as $n$ approaches infinity, the random variables
		$\sqrt{n}(S_n - \mu)$ converge in distribution to a normal $N(0, \sigma^2)$.
	\end{block}
	\vskip 2ex
	\begin{center}
		\fbox{\href{\wlnewdoc{basics-exercise-2.tex}}{%
		Clique aqui para abrir esse exerc\'icio no \wllogo{}}}
	\end{center}

	\begin{itemize}
		\item Dica: o comando para $\infty$ \'e \cmdbs{infty}.
		\item Uma vez que voc\^e tenha tentado, 
		\fbox{\href{\wlnewdoc{basics-exercise-2-solution.tex}}{%
		clique aqui para ver a solu\c{c}\~ao}}.
	\end{itemize}
\end{frame}

%%%%%%%%%%%%%%%%%%%%%%%%%%%%%%%%%%%%%%%%%%%%%%%%%%%%%%%%%%%%%%%%%%%%%%%%%%%%%%%
%%%%%%%%%%%%%%%%%%%%%%%%%%%%%%%%%%%%%%%%%%%%%%%%%%%%%%%%%%%%%%%%%%%%%%%%%%%%%%%
%%%%%%%%%%%%%%%%%%%%%%%%%%%%%%%%%%%%%%%%%%%%%%%%%%%%%%%%%%%%%%%%%%%%%%%%%%%%%%%
\begin{frame}{Final da Parte 1}
	\begin{itemize}
	\item Parabéns! Você já aprendeu como \ldots
		\begin{itemize}
		\item Compor textos em \LaTeX.
		\item Usar vários comandos diferentes.
		\item Tratar erros quando eles aparecem.
		\item Escrever equações matemáticas bonitas.
		\item Usar diversos ambientes diferentes.
		\item Carregar pacotes.
		\end{itemize}
		\item Isso é maravilhoso!
		\item Na Parte 2, veremos como usar \LaTeX{} para escrever documentos estruturados em seções, referências cruzadas, figuras, tabelas e bibliografias. Nos vemos lá!
	\end{itemize}
\end{frame}

\end{document}
